% course template
% https://extension.harvard.edu/wp-content/uploads/sites/8/2022/06/sample-evaluation-questionnaire.pdf
% latex template
% https://www.overleaf.com/project/67b3076423ccc422defa3c6c

\documentclass[letterpaper]{article}
\usepackage[top=3cm,bottom=3cm,left=2cm,right=2cm]{geometry}
\usepackage{textcomp,amsmath,longtable}
\usepackage{latexsym}
% Japanese translation
\usepackage{CJKutf8}
\usepackage[T1]{fontenc}
\usepackage[japanese]{babel}
\usepackage{hyperref}
\newcommand\textjp[1]{%
	\begin{CJK}{UTF8}{min}#1\end{CJK}}
\usepackage{graphicx}
\usepackage{fancyhdr}
\usepackage{booktabs}
\pagestyle{fancy}
%headers
%\lhead{\begin{picture}(0,0)\put(0,10){\includegraphics[width=2.5cm]{itm}}\end{picture}}



\chead{\LARGE \textsc{Course Evaluation Form \\ \textjp{コース評価フォーム}}\\

\small 
\textsc{ Tohoku University, Graduate School of Economics and Management} 
\textsc{\textjp{東北大学大学院経済学研究科}} \\
\textsc{ Graduate Student Association -- \textjp{院生会} } 
}
%\rhead{
%\textsc{Tecnológico Nacional de México}\\
%\textsc{Instituto Tecnológico de Morelia}\\
%\textsc{Departamento de Ingeniería Electrónica}\\}
\lfoot{Ver 1-2025/02/17}
%\rfoot{\begin{picture}(0,0)\put(-40,-25){\includegraphics[width=2.5cm]{itm}} \end{picture}}
\renewcommand{\headrulewidth}{0pt}
\headsep=45pt
%end headers----
%=-=-=-=-=-=-=-=

% begin questionnaires--
% Set up counters for questions and headings
\newcounter{hnumber}
\setcounter{hnumber}{0}
%\stepcounter{hnumber}
\newcounter{qnumber}
\setcounter{qnumber}{0}
\stepcounter{qnumber}
%=-=-=-=-=-=-=-=
% defining answer macro
\newcommand{\yesno}{{\Large ~$\Box$}}

%=-=-=-=-=-=-=-=%=-=-=-=-=-=-=-=
%% we now define the questions
\newcommand{\question}[1]{
	\hfill \relax \thehnumber.\theqnumber\hfill\hfill &
	\parbox[t]{8cm}{\raggedright #1} &{\small\yesno} & {\small\yesno} &{\small\yesno} &{\small\yesno}&{\small\yesno}\stepcounter{qnumber}\\
}

%=-=-=-=-=-=-=-=%=-=-=-=-=-=-=-=
%% we now define the headings of questions
\newcommand{\heading}[1]{
	\stepcounter{hnumber}
	\thehnumber &\multicolumn{4}{l}{\bf\textsf{#1}} \\ \midrule%
	}

%=-=-=-=-=-=-=-=
\newcommand{\divider}{\hline}

%=-=-=-=-=-=-=-=%=-=-=-=-=-=-=-=
% Document's content
\begin{document}

\centering

\vspace*{5px}

Instructor's name | \textjp{講師名} \rule{6cm}{0.5pt}\\
\vspace{5px}
Course’s name | \textjp{講座名} \rule{6cm}{0.5pt}\\
\vspace{5px}
Your Year and Major| \textjp{あなたの学年と専攻} \rule{6cm}{0.5pt}\\
%\vspace{5px}
%\textbf{Instructions:} Please mark the score given for each evaluation criterion below. \\
%\textjp{各評価基準について、以下の点数をマークしてください。}

\begin{longtable}{c p{8cm}ccccc}
% Main table's header
\textsc{No.} &\textsc{Description}&\footnotesize \textsc{Poor} & \footnotesize \textsc{Fair} & \footnotesize \textsc{Average}& \footnotesize \textsc{Good}&\footnotesize \textsc{Excellent}\\
% japanese
\textsc{} &\textsc{\textjp{内容}}&\footnotesize \textsc{\textjp{悪い}} & \footnotesize \textsc{\textjp{あまり良くない}} & \footnotesize \textsc{\textjp{普通}}& \footnotesize \textsc{\textjp{良い}}&\footnotesize \textsc{\textjp{とても良い}}\\
& & 1&2&3&4&5\\


%----------------------
% 1.
\heading{Course's content | \textjp{コースの内容について} (25 points)}
\question{The syllabus was useful to me \\ \textjp{シラバスが役に立った}}
\question{The course's structure is well organized \\ \textjp{コースの構成はよくまとまっている}}
\question{The course materials helped me understand the content \\ \textjp{教材のおかげで内容を理解できた}}
\question{The feedback on my homework was helpful \\ \textjp{宿題のフィードバックが役に立った} }
\question{The atmostphere of the course is good \\ \textjp{コースの雰囲気は良い}}

\hline
\vspace*{1px}
&\emph{Comments on the course's content} \\
&\emph{\textjp{コースの内容についてのコメント}} \\
\vspace{10px}

&&&&&Pts:&\rule{1.5cm}{0.5pt}/25\\
%----------------------
%2.
\setcounter{qnumber}{0}
\stepcounter{qnumber}
\heading{Instructor | \textjp{講師について} (25 points)}
\question{How would you rate your overall experience with the instructor? \\ \textjp{全体的な指導・サポートをどのように評価するか}}
\question{The instructor gave clear and easy-to-understand presentations \\ \textjp{わかりやすいプレゼンテーションだったか}}
\question{The instructor led class discussion/activities effectively \\ \textjp{ディスカッションや活動は効果的に進行されたか}}
\question{The instructor gave satisfactory answers to content questions \\ \textjp{講師は内容に関する質問に対して満足のいく回答をしたか}}
\question{The instructor's methods are useful for your study \\ \textjp{講師の教育方針は勉強に役に立ったか}}
\hline
\vspace*{1px}
&\emph{Comments on Instructor (what did the instructor do that you found helpful, what might the instructor have done differently to help you learn more):} \\
&\emph{\textjp{講師へのコメントを書いてください(講師が行ったことで役に立ったと感じたことは何ですか? また、さらに学びを深めるために講師がどのようにすれば良かったと思いますか)}} \\




\\&&&&&Pts: &\rule{1.5cm}{0.5pt} /25\\
\bottomrule
\end{longtable}

\small \noindent \bf Do you have any suggestions for improvements? (use backside)
\\
\small \noindent \bf \textjp{他にコメントや提案があったら、裏面を使って記入してください。}
\vspace{2cm}


\end{document}
