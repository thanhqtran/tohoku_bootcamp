\documentclass[10pt,a4paper]{article}
\usepackage[T1]{fontenc}

\usepackage{authblk}
\usepackage[right=1.25in,left=1.25in,top=1.1in,bottom=1.1in]{geometry}
% math tools
\usepackage{amsmath}
\usepackage{amsfonts}
\usepackage{amssymb}
\usepackage{amsthm}
\usepackage{mathtools}
\usepackage{mathrsfs}
% making hyperlinks blue
\usepackage[hidelinks]{hyperref}
\usepackage{xcolor}
\hypersetup{
	colorlinks,
	linkcolor={red!50!black},
	citecolor={blue!50!black},
	urlcolor={blue!80!black}
}
%bib also backref and blue
\usepackage{enotez}
\setenotez{backref}
\usepackage{float}
%enable footnotes' brackets
%\renewcommand*{\thefootnote}{(\arabic{footnote})}
%\renewcommand*{\theendnote}{(\arabic{endnote})}
%enable to collect footnotes as endnote
%\let\footnote = \endnote

%\usepackage{float}
\usepackage{ulem}
% creating nice table (\toprule, \midrule, \bottomrule)
\usepackage{booktabs}
\usepackage{threeparttable}

%make enumerate alphabetical instead of bullets
\usepackage{enumitem}
%\begin{enumerate}[label=(\alph*)] to use a), b), c) \end{enumerate}
%\begin{enumerate}[label=(\roman*)] \end{enumerate} to use i), ii), iii)

% tikz
\usepackage{tikz}
\usepackage{tikzscale}
\usetikzlibrary{arrows,calc, automata, patterns, positioning, shapes.geometric, decorations.pathreplacing,decorations.markings}

%to insert images side-by-side, change fig names
\usepackage[font=small,labelfont=bf,
justification=justified,
format=plain]{caption}
\captionsetup[figure]{name=Fig. ,labelsep=period}
\usepackage{subcaption}
\usepackage{graphicx}
\usepackage{rotating}


%----------bib manager---------%
%\usepackage[backend=bibtex,style=authoryear,natbib=true]{biblatex} 
\usepackage{natbib}
\bibliographystyle{apalike}
% Use the bibtex backend with the authoryear citation style (which resembles APA)
%to cite normally, use \textcite{}, and to cite with parentheses, use \parencite{}
\usepackage[autostyle=true]{csquotes} % Required to generate language-dependent quotes in the bibliography
%--offline
%\addbibresource{bibliography.bib} 
%--online
%\addbibresource[location=remote]{biblink}

%add plot
\usepackage{pgfplots}
\usepgfplotslibrary{groupplots}
\pgfplotsset{width=10cm,compat=1.9}

% color scheme
\newcommand{\red}[1]{\textcolor{red}{#1}}
\newcommand{\blue}[1]{\textcolor{blue}{#1}}
\newcommand{\green}[1]{\textcolor{green}{#1}}
\newcommand{\teal}[1]{\textcolor{teal}{#1}}

% quick maths
\newtheorem{theorem}{Theorem}[section]
\newtheorem{corollary}{Corollary}[theorem]
\newtheorem{lemma}[theorem]{Lemma}
\newtheorem{assumption}{Assumption}
\theoremstyle{definition}\newtheorem{definition}{Definition}
\newtheorem{prop}{Proposition}
\newtheorem{notation}{Notation}
\theoremstyle{definition}\newtheorem{fact}{Fact}
\theoremstyle{definition}\newtheorem{remark}{Remark}
\renewcommand\qedsymbol{$\blacksquare$}
\theoremstyle{definition}\newtheorem{ex}{Ex.}
\theoremstyle{definition}\newtheorem{project}{Project}
%problem/solution env

\theoremstyle{definition}\newtheorem{problem}{Problem}
\newenvironment{solution}{\begin{proof}[Solution]}{\end{proof}}
\theoremstyle{definition}\newtheorem{example}{Example}


%add frame for important stuff
\usepackage{mdframed}
\newenvironment{ftheorem}
{\begin{mdframed}\begin{theorem}}
		{\end{theorem}\end{mdframed}}
\newenvironment{fdefinition}
{\begin{mdframed}\begin{definition}}
		{\end{definition}\end{mdframed}}
\newenvironment{fprop}
{\begin{mdframed}\begin{prop}}
		{\end{prop}\end{mdframed}}
\newenvironment{fnotation}
{\begin{mdframed}\begin{notation}}
		{\end{notation}\end{mdframed}}



%--- to insert plain text---
%\texttt{code}

%convenience
%mathbb
\def\R{\mathbb R}
\def\N{\mathbb N}
\def\E{\mathbb E}
\def\P{\mathbb P}

%mathcal
\def\cP{\mathcal P}
\def\cS{\mathcal S}
\def\cX{\mathcal X}
\def\cY{\mathcal Y}
\def\cA{\mathcal A}
\def\cB{\mathcal B}
\def\cW{\mathcal W}

%mathbf
\def\1{\mathbf 1}
\def\0{\mathbf 0}
\def\A{\mathbf A}
\def\B{\mathbf B}
\def\C{\mathbf C}
\def\D{\mathbf D}
\def\E{\mathbf E}
\def\M{\mathbf M}
\def\N{\mathbb N}
\def\O{\mathbf O}
\def\P{\mathbf P}
\def\Q{\mathbf Q}
\def\R{\mathbb R}
\def\S{\mathbf S}
\def\I{\mathbf I}
\def\J{\mathbf J}
\def\T{\mathbf T}
\def\a{\mathbf a}
\def\b{\mathbf b}
\def\c{\mathbf c}
\def\e{\mathbf e}
\def\F{\mathbf F}
\def\G{\mathbf G}
\def\H{\mathbf H}
\def\h{\mathbf h}
\def\g{\mathbf g}
\def\m{\mathbf m}
\def\p{\mathbf p}
\def\q{\mathbf q}
\def\r{\mathbf r}
\def\s{\mathbf s}
\def\t{\mathbf t}
\def\u{\mathbf u}
\def\v{\mathbf v}
\def\U{\mathbf U}
\def\w{\mathbf w}
\def\x{\mathbf x}
\def\y{\mathbf y}
\def\Y{\mathbf Y}
\def\z{\mathbf z}
\def\Z{\mathbb Z}
\def\X{\mathbf X}
\def\V{\mathbf V}

% appendix appears in toc
\usepackage[titletoc]{appendix}



% ENUMERATE
\usepackage{enumitem}
% [label=(\alph*)] 
% [label=(\Alph*)]
%[label=(\roman*)]
%[label=(\arabic*)]

%enable footnotes' brackets
\renewcommand*{\thefootnote}{(\arabic{footnote})}
\renewcommand*{\theendnote}{(\arabic{endnote})}
%enable to collect footnotes as endnote
%\let\footnote = \endnote

\title{Math Refresher}
\author{Quang-Thanh Tran}
\begin{document}
	\maketitle
	
	\section{Calculus}
	\subsection{Derivatives}
	\begin{align}
		u(x,y) &= x^\alpha y^\beta, \\
		u(x,y) &= 2x + 3y, \\
		f(t) &= \frac{1}{e^t+e^{-t}}, \\
		f(t) &= e^{t/2}+e^{-t/2}, \\
		g(x) &= x^3(\ln x)^2, \\
		g(x) &= (\ln x+3x)^2, \\
		u(x,y) &= (x-2)^3(y-1), \\
		u(c) &= \frac{1}{1-\theta} (c^{1-\theta} - 1), \\
		\pi(q) &= (\bar{q}-q)q - \gamma q, \\
		c(q) &= b + a q - 0.5 q^2 + q^3, \\
		f(x) &= \frac{\sqrt{x}-2}{\sqrt{x}+1}, \\
		f(x) &=  \frac{x^2-1}{x^2+1}, \\
		f(x) &=  \frac{x^2+x+1}{x^2-x+1}
	\end{align}
	
	\subsection{Integration}
	\begin{align}
		&\int (3x^4+5x^2+2)dx, \\
		&\int \frac{(y-2)^2}{\sqrt{y}}dy, \\
		&\int_2^5 e^{2x}dx, \\
		&\int_{-2}^2 (x-x^3-x^5)dx 
	\end{align}
	by parts
	\begin{align}
		\int xe^x dx, \\
		\int (1/x)\ln x dx, \\
		\int x^3 e^{2x} dx, \\
		 \int xe^{-x} dx, \\
		 \int 3x e^{4x}dx
	\end{align}
	
	\section{Matrix Algebra}
	\subsection{Multiplication}
		\begin{align*}
			&\A = \begin{pmatrix}
				0 & -2 \\
				3 & 1 \\
			\end{pmatrix}  & \B = \begin{pmatrix}
				-1 & 4 \\
				1 & 5
			\end{pmatrix} \\
			&\C = \begin{pmatrix}
				8 & 3  & -2 \\
				1 & 0  & 4  
			\end{pmatrix} & \D = \begin{pmatrix}
				2 & -2 \\
				4 & 3 \\
				1 & -5
			\end{pmatrix} \\
			&\E= \begin{pmatrix}
				0 & 1  & 2 \\
				2 & 3  & 1 \\
				4 & -1 & 6 
			\end{pmatrix}, &
			\F = \begin{pmatrix}
				3 & 2  \\
				1 & 0 \\
				-1 & 1
			\end{pmatrix}, \\
			&\G = \begin{pmatrix}
				1 & 2  & -3 \\
				5 & 0  & 2  \\
				1 & -1 & 1  
			\end{pmatrix}, & 
			\H = \begin{pmatrix}
				3 & -1 & 2  \\
				4 & 2  & 5  \\
				2 & 0  & 3  
			\end{pmatrix}
		\end{align*}
	Multiply $\A\times \B$, $\C\times \D$, $\E\times \F$, $\G\times \H$. 
	\subsection{Determinants}	
	Find the determinants of
	\begin{align}
		\A &= \begin{pmatrix}
			1 & 4 \\ 2 & 8
		\end{pmatrix}, \\
		\B &= \begin{pmatrix}
			1 & 2 \\ 3 & 4
		\end{pmatrix}, \\
		\C &= \begin{pmatrix}
				8 & 1  & 3  \\
				4 & 0  & 1  \\
				6 & 0  & 3  
			\end{pmatrix}, \\
		\D &= \begin{pmatrix}
				1 & 2  & 3  \\
				4 & 7  & 5  \\
				3 & 6  & 9  
			\end{pmatrix}, \\
		\E &= \begin{pmatrix}
				1 & 2  & 3  \\
				8 & 9  & 4  \\
				7 & 6  & 5  
			\end{pmatrix}	
	\end{align}
	
	\section{Simple Optimization}
	Find the critical point by setting $f'(x) = 0$. If $f''(x) > 0$, it's a min, if $f''(x)<0$, it's a max.
	\begin{align}
		\max y(x) = \ln x - 5x, \\
		\min f(x) = e^x + e^{-2x}
	\end{align}
	
\end{document}